\documentclass[12pt]{article}

%\usepackage{algo}
\usepackage{tikz,fullpage,url,amssymb,amsmath,epsfig,color,xspace,alltt,mathtools}
\usetikzlibrary{shapes,chains,positioning}
\usepackage[pdftitle={CS 240 Assignment 2},%
pdfsubject={University of Waterloo, CS 240, Fall 2021},%
pdfauthor={MP}]{hyperref}
%\RequirePackage{pstricks,pst-node,pst-tree} % draw trees, requires using xetex
\newlength{\nodeLength}
\newcommand{\Node}{A}
\newcommand{\setnode}[1]{
	\settowidth{\nodeLength}{#1}
	\renewcommand{\Node}[1]{
		\Tcircle[name=#1]{\makebox[\nodeLength]{##1}}
	}
}
\setnode{99}

\newcommand{\ceil}[1]{\left\lceil #1 \right\rceil}
\newcommand{\floor}[1]{\left\lfloor #1 \right\rfloor}
\renewcommand{\thesubsection}{Problem \arabic{subsection}}

\begin{document}
	
	\begin{center}
		{\Large\bf Assignment 2 Problem 5}\\
		\vspace{3mm}
	\end{center}
	
	\definecolor{care}{rgb}{0,0,0}
	\def\question#1{\item[\bf #1.]}
	\def\part#1{\item[\bf #1)]}
	\newcommand{\pc}[1]{\mbox{\textbf{#1}}} % pseudocode
	
	
	
	%%%%%%%%%%%%%%%%%%%%%%%%%%%%%%%%%%%%%%%%%%%%%%%%%%%%%%%%%%%%%
	In this question, we generalize \textit{quickSelect1} to work on two input arrays. 
	Let the resulting algorithm be called \emph{quickSelect2Arrays(A,B,k)}.
	Arrays $A$ and $B$ are of size $n$ and $m$, respectively, and $k\in \{ 0,1,...,n+m-1 \}$. 
	Algorithm \emph{quickSelect2Arrays(A,B,k)} should return the item that would be in $C[k]$ if $C$ was the array resulting from merging arrays $A$ and $B$ and $C$ was sorted in non-decreasing order. 
	
	\bigskip
	
	\noindent Your algorithm \emph{quickSelect2Arrays(A,B,k)} must be in-place, i.e. only $O(1)$ additional space is allowed.
	Briefly and informally (one or two sentences) argue that the time complexity of your algorithm is the same as of \emph{quickSelect1}, i.e. $O(v)$ in the average case where $v$ is the total number of elements in $A$ and $B$, i.e. $v=n+m$.\\
	
	
	Solution:
	
	The pseudocode for quickSelect2Arrays(A,B,k):\\
	\\
	quickSelect2Arrays(A,B,k)\\
	A,B: two arrays\\
	k: the element of this postion that we want in the merging arrays.\\
	p1 $\gets$ choose-pivot(A)\\
	p2 $\gets$ choose-pivot(B)\\
	i1 $\gets$ partition(A,p1)\\
	i2 $\gets$ partition(B,p2)\\
	if k = 0 then\\
	\hphantom{1111} return min(A[0],B[0])\\
	else if min(i1,i2) $>$ k then\\
	\hphantom{1111} return quickSelect2Arrays(A[0,1,...,i1-1],B[0,1,...,i2-1],k)\\
	else if i1+i2+1 $<$ k then\\
	\hphantom{1111} return quickSelect2Arrays(A[i1+1,...,n-1],B[i2+1,...,m-1],k-(i1+1)-(i2+1))\\
	else if A[i1] = A[i2] then\\
	\hphantom{1111} return quickSelect2Arrays(A[0,1,...,i1-1],B[0,1,...,i2-1],k)\\
	else if A[i1] > A[i2] then\\
	\hphantom{1111} return quickSelect2Arrays(A[0,1,...,i1-1],B[0,1,...,m],k)
	
	
	
	%%%%%%%%%%%%%%%%%%%%%%%%%%%%%%%%%%%%%%%%%%%%%%%%%%%%%%%%%%%%%
\end{document}