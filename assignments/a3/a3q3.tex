\documentclass[12pt]{article}
%\usepackage{algpseudocode} 
%\usepackage{algo}
\usepackage{fullpage,url,amssymb,epsfig,color,xspace,tikz,amsmath}
\usetikzlibrary{shapes,positioning,calc,chains}
\usepackage[pdftitle={CS 240 Assignment 3},%
pdfsubject={University of Waterloo, CS 240, Fall 2021},%
pdfauthor={MP}]{hyperref}
\RequirePackage{pstricks,pst-node,pst-tree} % draw trees, requires using xetex
\newlength{\nodeLength}
\newcommand{\Node}{A}
\newcommand{\setnode}[1]{
	\settowidth{\nodeLength}{#1}
	\renewcommand{\Node}[1]{
		\Tcircle[name=#1]{\makebox[\nodeLength]{##1}}
	}
}
\setnode{99}

% \snode{ID}{NUMBER} becomes \node{ID}[item]{\ensuremath{NUMBER}}
\newcommand{\snode}[2]{\node(#1)[item]{\ensuremath{#2}}}

% \nodelabel{SUBSCRIPT} becomes \node[label]{\ensuremath{S_SUBSCRIPT}}
\newcommand{\nodelabel}[1]{\node[label]{\ensuremath{S_#1}}}

\newcommand{\quesbox}[2]{\begin{center} \framebox[.5\textwidth]{%
			\raisebox{-5mm}[0mm][#1]{\begin{minipage}[t]{.45\textwidth}%
					{\normalsize\sf #2}{\phantom{ans}}\end{minipage}}} \end{center}}
\newcommand{\ceil}[1]{\left\lceil#1\right\rceil}
\newcommand{\floor}[1]{\left\lfloor#1\right\rfloor}
\renewcommand{\thesubsection}{Problem \arabic{subsection}}
\definecolor{typo}{rgb}{0.75,0,0}
\definecolor{care}{rgb}{0,0,0}
\begin{document}
	
	\begin{center}
		{\Large\bf Assignment 3 Problem 3}\\
		\vspace{3mm}
	\end{center}
	
	\definecolor{care}{rgb}{0,0,0}
	\def\question#1{\item[\bf #1.]}
	\def\part#1{\item[\bf #1)]}
	\newcommand{\pc}[1]{\mbox{\textbf{#1}}} % pseudocode
	
	
	
	%%%%%%%%%%%%%%%%%%%%%%%%%%%%%%%%%%%%%%%%%%%%%%%%%%%%%%%%%%%%%
	%%% Problem 3
	
	Which of the following binary trees must have height $O(\log n)$?  Justify your answer.
	
	\begin{enumerate}
		\part{a} There is a constant $c > 0$ such that for all nodes $z$ in $T$, $z.left.height \leq z.right.height + c$.
		
		This tree does not have height $O(\log n)$. Since there is no restriction on the height of the left subtree of a node, therefore, we can have a tree like all nodes are added on the right for each node. And for such tree, the height is $\Theta(n)$
		
		\part{b} Every node $z$ that is not a leaf in $T$ has exactly two children.
		
		This tree does not have height $O(\log n)$. We can also provide an example that the tree has the height $\Theta(n)$. Considering a tree that except the first level(which only contains a root), all levels has two nodes, one is a leaf, and another one is not. Therefore, let h be the height of the tree, and n be the number of nodes, we have the relation: $2*(h-1)+1=n$ rearrange and take logarithm on both sides, we have $h-1=\log(n-1)$ which implies that $h \in \Theta(\log n)$
		
		\part{c} Given a BST $T$, let $N(z)$ be the number of nodes in the subtree rooted at $z$.
		If $z$ is an empty subtree, then $N(z) = 0$. \\
		There is a constant $0 < c \leq 1$ such that for every node $z$ in $T$, $N(z.left) \geq c\times N(z.right) - 1$ and $N(z.right) \geq c\times N(z.left) - 1$.
		
		This tree has height $O(\log n)$.\\
		We have the relation $N(left) \geq c\times N(right) - 1$ and $N(right) \geq c\times N(left) - 1$.\\
		If we know the N(left), then we can get $c\times N(left)-1 \leq N(right) \leq \frac{N(left)+1}{c}$\\
		Fix n, let h(n) be the max hight for n-node tree. We have the relation $h(n) = h(left) + 1$ if we assume that $N(left) \geq N(right)$\\
		Also, we have that $N(left) + N(right) + 1 = n$ and $N(right) = c\times N(left) -1$\\
		Simplifying, then we get $N(left)=\frac{n}{c+1}$ and $h(n) = h(\frac{n}{c+1})+1$\\
		Note that $1<c+1\leq 2$\\
		when n = 0, h(0) = -1 $\leq \log (n+1) = 0$\\
		when n = 1, h(1) = 0 $\leq \log (n+1) = \log 2 = 1$\\
		assume for $i < n$, we have $h(i) \leq \log (i+1)$\\
		$h(n) = 1 + h(\frac{n}{c+1}) \leq \log(\frac{n}{c+1}) + 1 = \log(\frac{n}{c+1}) + \log 2 = \log(\frac{2n}{c+1}) \in O(\log n)$\\
		By induction, we prove that for such BST with n nodes, the height is $O(\log n)$
	\end{enumerate}
	
	
	
	
	
\end{document}