\documentclass[12pt]{article}
%\usepackage{algpseudocode} 
%\usepackage{algo}
\usepackage{fullpage,url,amssymb,epsfig,color,xspace,tikz,amsmath}
\usetikzlibrary{shapes,positioning,calc,chains}
\usepackage[pdftitle={CS 240 Assignment 3},%
pdfsubject={University of Waterloo, CS 240, Fall 2021},%
pdfauthor={MP}]{hyperref}
\RequirePackage{pstricks,pst-node,pst-tree} % draw trees, requires using xetex
\newlength{\nodeLength}
\newcommand{\Node}{A}
\newcommand{\setnode}[1]{
	\settowidth{\nodeLength}{#1}
	\renewcommand{\Node}[1]{
		\Tcircle[name=#1]{\makebox[\nodeLength]{##1}}
	}
}
\setnode{99}

% \snode{ID}{NUMBER} becomes \node{ID}[item]{\ensuremath{NUMBER}}
\newcommand{\snode}[2]{\node(#1)[item]{\ensuremath{#2}}}

% \nodelabel{SUBSCRIPT} becomes \node[label]{\ensuremath{S_SUBSCRIPT}}
\newcommand{\nodelabel}[1]{\node[label]{\ensuremath{S_#1}}}

\newcommand{\quesbox}[2]{\begin{center} \framebox[.5\textwidth]{%
			\raisebox{-5mm}[0mm][#1]{\begin{minipage}[t]{.45\textwidth}%
					{\normalsize\sf #2}{\phantom{ans}}\end{minipage}}} \end{center}}
\newcommand{\ceil}[1]{\left\lceil#1\right\rceil}
\newcommand{\floor}[1]{\left\lfloor#1\right\rfloor}
\renewcommand{\thesubsection}{Problem \arabic{subsection}}
\definecolor{typo}{rgb}{0.75,0,0}
\definecolor{care}{rgb}{0,0,0}
\begin{document}
	
	\begin{center}
		{\Large\bf Assignment 3 Problem 5}\\
		\vspace{3mm}
	\end{center}
	
	\definecolor{care}{rgb}{0,0,0}
	\def\question#1{\item[\bf #1.]}
	\def\part#1{\item[\bf #1)]}
	\newcommand{\pc}[1]{\mbox{\textbf{#1}}} % pseudocode
	
	
	
	%%%%%%%%%%%%%%%%%%%%%%%%%%%%%%%%%%%%%%%%%%%%%%%%%%%%%%%%%%%%%
	%%% Problem 5
	
	Consider the list of keys:
	\[
	[1\ 2\ 3\ 4\ 5\ 6\ 7\ 8\ 9\ 10]
	\]
	and assume we perform the following searches:
	\begin{center}
		8, 6, 3, 7, 5$^\star$, 5, 2, 6, 2, 8$^\star$, 3, 9, 3, 8$^\star$ 
	\end{center}
	\begin{enumerate}
		\part{a} Using the move-to-front heuristic, give the list ordering after the starred ($\star$) searches are performed. 
		Additionally, record the number of comparisons between keys after each search, as well as, the total number of comparisons.\\
		The ordering after the 5$^\star$:
		\[
		[5\ 7\ 3\ 6\ 8\ 1\ 2\ 4\ 9\ 10]
		\]\\
		The ordering after the first 8$^\star$:
		\[
		[8\ 2\ 6\ 5\ 7\ 3\ 1\ 4\ 9\ 10]
		\]\\
		The ordering after the second 8$^\star$:
		\[
		[8\ 3\ 9\ 2\ 6\ 5\ 7\ 1\ 4\ 10]
		\]
		\begin{center}
			\begin{tabular}{|c|c|c|c|c|c|c|c|c|c|c|c|c|c||c|} \hline
				8 &  6 &  3 &  7 &  5 & 5 & 2 & 6 & 2 & 8 & 3 & 9 & 3 & 8  & Total \\ \hline
				8 & 7 & 5 & 8 & 8 & 1 & 7 & 5 & 2 & 6 & 6 & 9 & 2 & 3 & 77\\ \hline
			\end{tabular}
		\end{center}
		
		\part{b} Repeat part (a), using the transpose heuristic instead of the move-to-front heuristic.
		
		The ordering after the 5$^\star$:
		\[
		[1\ 3\ 2\ 4\ 5\ 6\ 7\ 8\ 9\ 10]
		\]\\
		The ordering after the first 8$^\star$:
		\[
		[2\ 1\ 3\ 5\ 6\ 4\ 8\ 7\ 9\ 10]
		\]\\
		The ordering after the second 8$^\star$:
		\[
		[3\ 2\ 1\ 5\ 6\ 8\ 4\ 9\ 7\ 10]
		\]
		\begin{center}
			\begin{tabular}{|c|c|c|c|c|c|c|c|c|c|c|c|c|c||c|} \hline
				8 &  6 &  3 &  7 &  5 & 5 & 2 & 6 & 2 & 8 & 3 & 9 & 3 & 8  & Total \\ \hline
				8 & 6 & 3 & 8 & 6 & 5 & 3 & 6 & 2 & 8 & 3 & 9 & 2 & 7 & 76\\ \hline
			\end{tabular}
		\end{center}
		
		\part{c} Another heuristic is \textit{move-to-front2 (MTF2)} that is similar to
		\textit{move-to-front (MTF)} except that when an element is found at position $i$ it is moved to position $\left \lfloor \frac{i}{2} \right \rfloor$.  
		Repeat part (a), using this heuristic.  
		
		The ordering after the 5$^\star$:
		\[
		[1\ 3\ 2\ 5\ 7\ 6\ 8\ 4\ 9\ 10]
		\]\\
		The ordering after the first 8$^\star$:
		\[
		[2\ 1\ 6\ 8\ 5\ 3\ 7\ 4\ 9\ 10]
		\]\\
		The ordering after the second 8$^\star$:
		\[
		[2\ 3\ 8\ 1\ 6\ 9\ 5\ 7\ 4\ 10]
		\]
		\begin{center}
			\begin{tabular}{|c|c|c|c|c|c|c|c|c|c|c|c|c|c||c|} \hline
				8 & 6 & 3 & 7 & 5 & 5 & 2 & 6 & 2 & 8 & 3 & 9 & 3 & 8 & Total \\ \hline
				8 & 7 & 3 & 8 & 8 & 4 & 4 & 6 & 2 & 7 & 6 & 9 & 3 & 6 &  81\\ \hline
			\end{tabular}
		\end{center}
		
	\end{enumerate}
	
\end{document}