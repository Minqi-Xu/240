\documentclass[12pt]{article}
\usepackage{algpseudocode} 
\usepackage{fullpage,url,amssymb,epsfig,color,xspace,tikz,amsmath}
\usetikzlibrary{shapes,positioning,calc,chains}
\usepackage[pdftitle={CS 240 Assignment 4},%
pdfsubject={University of Waterloo, CS 240, Fall 2021},%
pdfauthor={MP}]{hyperref}
\RequirePackage{pstricks,pst-node,pst-tree} % draw trees, requires using xetex
\newlength{\nodeLength}
\newcommand{\Node}{A}
\newcommand{\setnode}[1]{
	\settowidth{\nodeLength}{#1}
	\renewcommand{\Node}[1]{
		\Tcircle[name=#1]{\makebox[\nodeLength]{##1}}
	}
}
\setnode{99}

% \snode{ID}{NUMBER} becomes \node{ID}[item]{\ensuremath{NUMBER}}
\newcommand{\snode}[2]{\node(#1)[item]{\ensuremath{#2}}}

% \nodelabel{SUBSCRIPT} becomes \node[label]{\ensuremath{S_SUBSCRIPT}}
\newcommand{\nodelabel}[1]{\node[label]{\ensuremath{S_#1}}}

\newcommand{\quesbox}[2]{\begin{center} \framebox[.5\textwidth]{%
			\raisebox{-5mm}[0mm][#1]{\begin{minipage}[t]{.45\textwidth}%
					{\normalsize\sf #2}{\phantom{ans}}\end{minipage}}} \end{center}}
\newcommand{\ceil}[1]{\left\lceil#1\right\rceil}
\newcommand{\floor}[1]{\left\lfloor#1\right\rfloor}
\renewcommand{\thesubsection}{Problem \arabic{subsection}}
\definecolor{typo}{rgb}{0.75,0,0}
\definecolor{care}{rgb}{0,0,0}
\begin{document}
	
	\begin{center}
		
		{\Large\bf Assignment 4 Problem 4}\\
		\vspace{3mm}
	\end{center}
	
	\definecolor{care}{rgb}{0,0,0}
	\def\question#1{\item[\bf #1.]}
	\def\part#1{\item[\bf #1)]}
	\newcommand{\pc}[1]{\mbox{\textbf{#1}}} % pseudocode
	
	
	%%%%%%%%%%%%%%%%%%%%%%%%%%%%%%%%%%%%%%%%%%%%%%%%%%%%%%%%%%%%%
	%%% Problem 4
	
	For all parts of this question, use the convention that each internal node of a quadtree has exactly four children, corresponding to regions NE, NW, SW and SE, in that order.  Also, as stated in Module 8 Slide 6, the bounding box must be the smallest box containing all points where the width and height are a power of 2.
	
	\begin{itemize}
		\part{a} Give three 2-dimensional points such that the corresponding quadtree has height at least 50.
		Give the $(x,y)$ coordinates of the three points and show the shape of the quadtree; i.e. you do not need to show every level in the quadtree but show enough so the missing levels can be inferred. (Do not give the plane partitions.)
		
		\part{b} One application of quadtrees is image compression.  An image (picture) is recursively
		divided into quadrants until the entire quadrant is only one colour.  Using this rule, draw the quadtree
		of the following image.  There are only three colours (shades of grey). For the leaves of the quad tree,
		use 1 to denote the lightest shade, 2 for the middle shade and 3 for the darkest shade of grey.
		
		%		\hfill\includegraphics{qtimage.jpg}\hspace*{\fill}
		
		\part{c} Another application is to compare two images.  Given two black and white images (i.e. each pixel of the image is either 0 or 1)
		each of size $2^k \times 2^k$ stored as quadtrees, give an algorithm for the Intersection operation: if corresponding pixels in both images is 1, then their intersection is 1; otherwise 0.
		
		Runtime analysis is not required but your algorithm should be as efficient as possible; i.e. marks may be deducted for terribly inefficient implementations.
		
	\end{itemize}
	
	
	solution:
	\begin{enumerate}
		\part{a} 
		
		\part{b} 
		
		\part{c} Given two images A and B which are two quadtree, then\\
		We do the tranverse on both quadtree at the same time. Compare each nodes of the quadtree(starting from root), if for A it is not a leaf, then we check B, if for B it is a leaf, then returns 0,if not then we check recursively on its children. When checking its children, if any of children returns 0, then we returns 0 immediately, if no children returns 0, we return 1. If for A it is a leaf, then we check B, if for B it is not a leaf or it is a leaf but with a different number, then returns 0, otherwise, we just return 1..
		
		
	\end{enumerate}
	
\end{document}