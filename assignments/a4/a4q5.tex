\documentclass[12pt]{article}
\usepackage{algpseudocode} 
\usepackage{fullpage,url,amssymb,epsfig,color,xspace,tikz,amsmath}
\usetikzlibrary{shapes,positioning,calc,chains}
\usepackage[pdftitle={CS 240 Assignment 4},%
pdfsubject={University of Waterloo, CS 240, Fall 2021},%
pdfauthor={MP}]{hyperref}
\RequirePackage{pstricks,pst-node,pst-tree} % draw trees, requires using xetex
\newlength{\nodeLength}
\newcommand{\Node}{A}
\newcommand{\setnode}[1]{
	\settowidth{\nodeLength}{#1}
	\renewcommand{\Node}[1]{
		\Tcircle[name=#1]{\makebox[\nodeLength]{##1}}
	}
}
\setnode{99}

% \snode{ID}{NUMBER} becomes \node{ID}[item]{\ensuremath{NUMBER}}
\newcommand{\snode}[2]{\node(#1)[item]{\ensuremath{#2}}}

% \nodelabel{SUBSCRIPT} becomes \node[label]{\ensuremath{S_SUBSCRIPT}}
\newcommand{\nodelabel}[1]{\node[label]{\ensuremath{S_#1}}}

\newcommand{\quesbox}[2]{\begin{center} \framebox[.5\textwidth]{%
			\raisebox{-5mm}[0mm][#1]{\begin{minipage}[t]{.45\textwidth}%
					{\normalsize\sf #2}{\phantom{ans}}\end{minipage}}} \end{center}}
\newcommand{\ceil}[1]{\left\lceil#1\right\rceil}
\newcommand{\floor}[1]{\left\lfloor#1\right\rfloor}
\renewcommand{\thesubsection}{Problem \arabic{subsection}}
\definecolor{typo}{rgb}{0.75,0,0}
\definecolor{care}{rgb}{0,0,0}
\begin{document}
	
	\begin{center}
		
		{\Large\bf Assignment 4 Problem 5}\\
		\vspace{3mm}
	\end{center}
	
	\definecolor{care}{rgb}{0,0,0}
	\def\question#1{\item[\bf #1.]}
	\def\part#1{\item[\bf #1)]}
	\newcommand{\pc}[1]{\mbox{\textbf{#1}}} % pseudocode
	
	
	%%%%%%%%%%%%%%%%%%%%%%%%%%%%%%%%%%%%%%%%%%%%%%%%%%%%%%%%%%%%%
	%%% Problem 5	
	\begin{enumerate}
		\part{a}  Draw the kd-tree representing the set of 2D points 
		
		$S = \{ p_1,...,p_8 \} = \{(1,1), (2,2), (3,3), (4,4),(5,5),(6,6), (7,7), (8,8)  \}$.
		
		\part{b} Give an algorithm using pseudocode for finding a point with the smallest x-coordinate in a kd-tree storing 2D points. Your algorithm should be as efficient as possible.  State and explain the worst case running time of your algorithm. 
		
	\end{enumerate}
	
	
	solution:
	\begin{enumerate}
		\part{a}
		
		\part{b}
		
	\end{enumerate}
	
\end{document}