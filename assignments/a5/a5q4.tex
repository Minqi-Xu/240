\documentclass[12pt]{article}
\usepackage{algpseudocode} 
\usepackage{fullpage,url,amssymb,epsfig,color,xspace,tikz,amsmath}
\usetikzlibrary{shapes,positioning,calc,chains}
\usepackage[pdftitle={CS 240 Assignment 5},%
pdfsubject={University of Waterloo, CS 240, Fall 2021},%
pdfauthor={MP}]{hyperref}
\RequirePackage{pstricks,pst-node,pst-tree} % draw trees, requires using xetex
\newlength{\nodeLength}
\newcommand{\Node}{A}
\newcommand{\setnode}[1]{
	\settowidth{\nodeLength}{#1}
	\renewcommand{\Node}[1]{
		\Tcircle[name=#1]{\makebox[\nodeLength]{##1}}
	}
}
\setnode{99}

% \snode{ID}{NUMBER} becomes \node{ID}[item]{\ensuremath{NUMBER}}
\newcommand{\snode}[2]{\node(#1)[item]{\ensuremath{#2}}}

% \nodelabel{SUBSCRIPT} becomes \node[label]{\ensuremath{S_SUBSCRIPT}}
\newcommand{\nodelabel}[1]{\node[label]{\ensuremath{S_#1}}}

\newcommand{\quesbox}[2]{\begin{center} \framebox[.5\textwidth]{%
			\raisebox{-5mm}[0mm][#1]{\begin{minipage}[t]{.45\textwidth}%
					{\normalsize\sf #2}{\phantom{ans}}\end{minipage}}} \end{center}}
\newcommand{\ceil}[1]{\left\lceil#1\right\rceil}
\newcommand{\floor}[1]{\left\lfloor#1\right\rfloor}
\renewcommand{\thesubsection}{Problem \arabic{subsection}}
\definecolor{typo}{rgb}{0.75,0,0}
\definecolor{care}{rgb}{0,0,0}
\begin{document}
	
	\begin{center}
		{\Large\bf Assignment 5 Problem 4}\\
		\vspace{3mm}
	\end{center}
	
	\definecolor{care}{rgb}{0,0,0}
	\def\question#1{\item[\bf #1.]}
	\def\part#1{\item[\bf #1)]}
	\newcommand{\pc}[1]{\mbox{\textbf{#1}}} % pseudocode
	
	\begin{enumerate}
		\part{a} Construct the last occurrence function $L$ for pattern $P=adobodoa$  where $\Sigma = {a,b,c,d,o,t}$.
		
		\part{b} Trace the search for $P$ in $T=dotadotadotdotadobodoadot$ using the Boyer-Moore algorithm.
		Indicate in a table such as Table \ref{bmtable} which characters of $P$ were compared with which characters of $T$.
		Follow the example from the lecture video in Module 9.
		Place each character of $P$ in the column of the compared-to character of $T$. 
		Put brackets around the character if they are known to match from the previous step (similar to the examples in the slides). 
		Use a new row when sliding the pattern. You may not need all rows in the table.  Add more rows to the table if you need more.
		
		
		\part{c} For any $m\geq 1$ and any $n\geq m$, give a pattern $P$ and a text $T$ such that the Boyer-Moore algorithm looks at exactly $\Theta(n/m)$ characters.  
		Justify your answer.
		
	\end{enumerate}
	
	
\end{document}