\documentclass[12pt]{article}
\usepackage{algpseudocode} 
\usepackage{fullpage,url,amssymb,epsfig,color,xspace,tikz,amsmath}
\usetikzlibrary{shapes,positioning,calc,chains}
\usepackage[pdftitle={CS 240 Assignment 5},%
pdfsubject={University of Waterloo, CS 240, Fall 2021},%
pdfauthor={MP}]{hyperref}
\RequirePackage{pstricks,pst-node,pst-tree} % draw trees, requires using xetex
\newlength{\nodeLength}
\newcommand{\Node}{A}
\newcommand{\setnode}[1]{
	\settowidth{\nodeLength}{#1}
	\renewcommand{\Node}[1]{
		\Tcircle[name=#1]{\makebox[\nodeLength]{##1}}
	}
}
\setnode{99}

% \snode{ID}{NUMBER} becomes \node{ID}[item]{\ensuremath{NUMBER}}
\newcommand{\snode}[2]{\node(#1)[item]{\ensuremath{#2}}}

% \nodelabel{SUBSCRIPT} becomes \node[label]{\ensuremath{S_SUBSCRIPT}}
\newcommand{\nodelabel}[1]{\node[label]{\ensuremath{S_#1}}}

\newcommand{\quesbox}[2]{\begin{center} \framebox[.5\textwidth]{%
			\raisebox{-5mm}[0mm][#1]{\begin{minipage}[t]{.45\textwidth}%
					{\normalsize\sf #2}{\phantom{ans}}\end{minipage}}} \end{center}}
\newcommand{\ceil}[1]{\left\lceil#1\right\rceil}
\newcommand{\floor}[1]{\left\lfloor#1\right\rfloor}
\renewcommand{\thesubsection}{Problem \arabic{subsection}}
\definecolor{typo}{rgb}{0.75,0,0}
\definecolor{care}{rgb}{0,0,0}
\begin{document}
	
	\begin{center}
		{\Large\bf University of Waterloo}\\
		\vspace{3mm}
		{\Large\bf CS240 Fall 2021}\\
		\vspace{2mm}
		{\Large\bf Assignment 5}\\
		\vspace{3mm}
		\textbf{Due Date: Wednesday, Dec 1 at 5:00pm}
	\end{center}
	
	\definecolor{care}{rgb}{0,0,0}
	\def\question#1{\item[\bf #1.]}
	\def\part#1{\item[\bf #1)]}
	\newcommand{\pc}[1]{\mbox{\textbf{#1}}} % pseudocode
	
	The integrity of the grade you receive in this course is very important to you and the University of Waterloo.  
	As part of every assessment in this course you must read and sign an Academic Integrity Declaration before you start working on the assessment and submit it \textbf{before the deadline of December 1st} along with your answers to the assignment; i.e. \textbf{read, sign and submit A05-AcInDe.txt now or as soon as possible}.
	The agreement will indicate what you must do to ensure the integrity of your grade.
	If you are having difficulties with the assignment, course staff are there to help (provided it isn't last minute). \\
	~\\
	\textbf{The Academic Integrity Declaration must be signed and submitted on time or the assessment will not be marked.} \\
	~\\
	Please read \url{http://www.student.cs.uwaterloo.ca/~cs240/f21/guidelines.pdf} for guidelines on submission.  
	\textbf{Each question must be submitted individually to MarkUs as a PDF} with the corresponding file names: 
	a5q1.pdf, a5q2.pdf, ... , a5q5.pdf . \\
	~\\
	It is a good idea to submit questions as you go so you aren't trying to create several PDF files at the last minute. \\
	~\\
	\textbf{Late Policy:} Assignments are due at 5:00pm.
	To accommodate any small time differences with our submission server, there is a grace period of 5 minutes; i.e. we will accept assignments until 5:05pm without penalty.  
	Assignments submitted after 5:05pm will incur a penalty of 1 mark per minute to a maximum of 10 marks.   \\
	\textbf{Assignments submitted after 5:15pm will not be accepted} but may be reviewed (by request) for feedback purposes only. \\  
	~\\
	
	%%%%%%%%%%%%%%%%%%%%%%%%%%%%%%%%%%%%%%%%%%%%%%%%%%%%%%%%%%%%%
	%%% Problem 1
	\subsection{Range Trees [3+1+3+3=10 marks]}
	
	\begin{itemize}
		\part{a} Draw a 2-dimensional range tree of minimal height for the following set of points:
		\begin{center}
			$\{(7, 88), (12, 19), (22, 33), (27, 29), (28, 9), (31, 99), (42, 66)\}$ 
		\end{center}
		
		\part{b} Suppose a two dimensional range tree data structure stores $n$ points, and that the BST ordered by $x$-coordinates is perfect, i.e., every level is completely filled.
		Give an exact closed form formula in terms of $n$ for the sum of the number of nodes in the $x$-ordered BST plus the total number of nodes in all $y$-ordered BSTs.
		
		\part{c} Assume that we have a set of $n$ numbers (not necessarily integers) and we are interested only in counting the number of points that lie in a range rather than in reporting all of them.
		Describe how a 1-dimensional range tree (i.e., a balanced BST) can be modified such that a range counting query can be performed in $O(\log n)$ time (independent of $s$).  
		Briefly justify that your algorithm is within the expected runtime.
		
		\part{d} Next, consider the 2-dimensional case where we have a set of $n$ 2-dimensional points.
		Given a query rectangle $R$, we only want to find the number of points inside $R$, not the points themselves.
		Explain how to modify the Range Tree data structure and the search algorithm such that counting queries can be performed in $O((\log n)^2))$ time.  
		Briefly justify that your algorithm meets the runtime requirement.
		
	\end{itemize}
	
	%%%%%%%%%%%%%%%%%%%%%%%%%%%%%%%%%%%%%%%%%%%%%%%%%%%%%%%%%%%%%
	%%% Problem 2
	\subsection{Rabin-Karp [2+5+3=10 marks]}
	
	A deranged instructor once posed the following question on a final exam: \\
	\textbf{Note: this is not the problem you are asked to solve.} \\
	~\\
	Given a text $T$ of length $n$ and an array $P$ of size $\ell$ of patterns where each $P[i]$, for $0 \leq i < \ell$, is a pattern, i.e. an array of characters, where:
	\begin{itemize}
		\item the length of $P[0]$ is $m$
		\item the length of $P[k]$ is $(k+1)m$ for $k \geq 0$ 
		\item $P[k-1]$ is a prefix of $P[k]$ for all $k > 0$
		\item $n$ is larger than the length of the longest pattern $P[\ell-1]$
	\end{itemize}
	
	State how to modify the Rabin-Karp algorithm so that it returns $(i,k)$ where $k$ is maximal such that $P[k]$ is found in $T$; i.e. $P[k+1]$ is not found in $T$. 
	The returned $i$ is the position in the text where pattern $P[k]$ is found.
	For simplicity, you may assume that $\Sigma = \{0, 1, 2, \ldots, 9\}$ and the hash function is the standard one described on Slide 16 of Module 9. \\
	~\\
	\textbf{One possible solution is to do the following:} \\
	~\\
	In the pre-processing step, create an array $V$ of hash values for each pattern (of increasing length) where $V[i]$ is the hash value for $P[i]$. \\
	~\\
	The algorithm then uses a binary search approach on the pattern hash value array to find the longest pattern found in the text.  
	For the initial step, let $b =$ the midpoint of $(0, \ell)$, then the Rabin-Karp algorithm is applied using hash value $V[b]$ and its corresponding pattern. \\
	If the pattern is found in the text, then a new pattern of $V$ is chosen at the midpoint of $(b+1, \ell)$;
	otherwise the pattern of $V$ at the midpoint of $(0, b-1)$ is chosen.  
	The Rabin-Karp algorithm is then applied again with this new pattern.
	The algorithm continues in this way until the maximal length pattern found in $T$ is determined.
	You may assume the appropriate values are returned (these details are not important for this question). \\
	~\\
	\textbf{The questions for you to answer are below.}
	
	\begin{enumerate}
		\part{a} Give an algorithm to efficiently create array $V$ and briefly justify the runtime.
		
		For parts b) and c) below, do not include the pre-processing time from part a).
		
		\part{b} Analyze the algorithm given and give the following runtimes:
		\begin{itemize}
			\item Best-case runtime
			\item Worst-case runtime
			\item Best-case expected runtime
			\item Worst-case expected runtime
		\end{itemize}
		Explain how the runtimes were determined. \\
		Note: The runtime function parameters are: $n$, $m$ and $\ell$.
		
		\part{c} Suppose the solution is modified so that if a pattern corresponding to $V[j]$ is found, Rabin-Karp is then executed using $V[j+1]$ and its corresponding pattern.  
		If this pattern is not found, we can conclude that the pattern for $V[j]$ is maximal and terminate.  
		Otherwise, the algorithm continues as originally stated. \\
		Does this change either of the runtimes from part b)?  Briefly justify your answer and give the new runtime(s) if they differ from the previous part. \\
		
	\end{enumerate}
	
	%%%%%%%%%%%%%%%%%%%%%%%%%%%%%%%%%%%%%%%%%%%%%%%%%%%%%%%%%%%%%
	%%% Problem 3
	\subsection{KMP [3+3=6 marks]} 
	
	\begin{enumerate}
		\part{a} Compute the failure array for the pattern $P=$\texttt{ababac}.
		
		\part{b}
		Show how to search for pattern $P=$\texttt{ababac} in the text $T=$\texttt{abcaabaabababacabcaa} using the KMP algorithm.
		Indicate in a table such as Table \ref{kmp} which characters of $P$ were compared with which characters of $T$. 
		Follow the example on slide~27 in module~8.
		Place each character of $P$ in the column of the compared-to character of $T$.  
		Put brackets around the character if an actual comparison was not performed. 
		You may not need all space in the table.
		
	\end{enumerate}
	\begin{table}
		\Large{
			\begin{center}
				\begin{tabular}{|c|c|c|c|c|c|c|c|c|c|c|c|c|c|c|c|c|c|c|c|}
					\hline
					a&b&c&a&a&b&a&a&b&a&b&a&b&a&c&a&b&c&a&a\\
					\hline
					\hline
					&&&&&&&&&&&&&&&&&&&\\
					\hline
					&&&&&&&&&&&&&&&&&&&\\
					\hline
					&&&&&&&&&&&&&&&&&&&\\
					\hline
					&&&&&&&&&&&&&&&&&&&\\
					\hline
					&&&&&&&&&&&&&&&&&&&\\
					\hline
					&&&&&&&&&&&&&&&&&&&\\
					\hline
					&&&&&&&&&&&&&&&&&&&\\
					\hline
					&&&&&&&&&&&&&&&&&&&\\
					\hline
					&&&&&&&&&&&&&&&&&&&\\
					\hline
					&&&&&&&&&&&&&&&&&&&\\
					\hline
				\end{tabular}
		\end{center}}
		\caption{Table for KMP problem.}\label{kmp}
	\end{table}
	
	%%%%%%%%%%%%%%%%%%%%%%%%%%%%%%%%%%%%%%%%%%%%%%%%%%%%%%%%%%%%%
	%%% Problem 4
	
	\subsection{Boyer-Moore [2+3+3=8 marks]}
	\begin{enumerate}
		\part{a} Construct the last occurrence function $L$ for pattern $P=adobodoa$  where $\Sigma = {a,b,c,d,o,t}$.
		
		\part{b} Trace the search for $P$ in $T=dotadotadotdotadobodoadot$ using the Boyer-Moore algorithm.
		Indicate in a table such as Table \ref{bmtable} which characters of $P$ were compared with which characters of $T$.
		Follow the example from the lecture video in Module 9.
		Place each character of $P$ in the column of the compared-to character of $T$. 
		Put brackets around the character if they are known to match from the previous step (similar to the examples in the slides). 
		Use a new row when sliding the pattern. You may not need all rows in the table.  Add more rows to the table if you need more.
		\begin{table}[ht]
			\Large{
				\begin{center}
					\begin{tabular}{|c|c|c|c|c|c|c|c|c|c|c|c|c|c|c|c|c|c|c|c|c|c|c|c|c|}
						\hline
						d&o&t&a&d&o&t&a&d&o&t&d&o&t&a&d&o&b&o&d&o&a&d&o&t\\
						\hline
						\hline
						&&&&&&&&&&&&&&&&&&&&&&&&\\
						\hline
						&&&&&&&&&&&&&&&&&&&&&&&&\\
						\hline
						&&&&&&&&&&&&&&&&&&&&&&&&\\
						\hline
						&&&&&&&&&&&&&&&&&&&&&&&&\\
						\hline
						&&&&&&&&&&&&&&&&&&&&&&&&\\
						\hline
						&&&&&&&&&&&&&&&&&&&&&&&&\\
						\hline
						&&&&&&&&&&&&&&&&&&&&&&&&\\
						\hline
						&&&&&&&&&&&&&&&&&&&&&&&&\\
						\hline
						&&&&&&&&&&&&&&&&&&&&&&&&\\
						\hline
					\end{tabular}
			\end{center}}
			\caption{Table for Boyer-Moore problem.}\label{bmtable}
		\end{table}
		
		\part{c} For any $m\geq 1$ and any $n\geq m$, give a pattern $P$ and a text $T$ such that the Boyer-Moore algorithm looks at exactly $\Theta(n/m)$ characters.  
		Justify your answer.
		
	\end{enumerate}
	
	\subsection{Suffix Trees and Arrays [3+3=6 marks]}
	\begin{enumerate}
		\part{a} Draw the suffix tree corresponding to the text $T =abracadabra$.
		\part{b} Construct suffix array for $S = balbes$.
	\end{enumerate}
	%%% END OF QUESTIONS %%%
\end{document}