\documentclass[12pt]{article}
\usepackage{algpseudocode} 
\usepackage{fullpage,url,amssymb,epsfig,color,xspace,tikz,amsmath}
\usetikzlibrary{shapes,positioning,calc,chains}
\usepackage[pdftitle={CS 240 Assignment 5},%
pdfsubject={University of Waterloo, CS 240, Fall 2021},%
pdfauthor={MP}]{hyperref}
\RequirePackage{pstricks,pst-node,pst-tree} % draw trees, requires using xetex
\newlength{\nodeLength}
\newcommand{\Node}{A}
\newcommand{\setnode}[1]{
	\settowidth{\nodeLength}{#1}
	\renewcommand{\Node}[1]{
		\Tcircle[name=#1]{\makebox[\nodeLength]{##1}}
	}
}
\setnode{99}

% \snode{ID}{NUMBER} becomes \node{ID}[item]{\ensuremath{NUMBER}}
\newcommand{\snode}[2]{\node(#1)[item]{\ensuremath{#2}}}

% \nodelabel{SUBSCRIPT} becomes \node[label]{\ensuremath{S_SUBSCRIPT}}
\newcommand{\nodelabel}[1]{\node[label]{\ensuremath{S_#1}}}

\newcommand{\quesbox}[2]{\begin{center} \framebox[.5\textwidth]{%
			\raisebox{-5mm}[0mm][#1]{\begin{minipage}[t]{.45\textwidth}%
					{\normalsize\sf #2}{\phantom{ans}}\end{minipage}}} \end{center}}
\newcommand{\ceil}[1]{\left\lceil#1\right\rceil}
\newcommand{\floor}[1]{\left\lfloor#1\right\rfloor}
\renewcommand{\thesubsection}{Problem \arabic{subsection}}
\definecolor{typo}{rgb}{0.75,0,0}
\definecolor{care}{rgb}{0,0,0}
\begin{document}
	
	\begin{center}
		{\Large\bf Assignment 5 Problem 1}\\
		\vspace{3mm}
	\end{center}
	
	\definecolor{care}{rgb}{0,0,0}
	\def\question#1{\item[\bf #1.]}
	\def\part#1{\item[\bf #1)]}
	\newcommand{\pc}[1]{\mbox{\textbf{#1}}} % pseudocode
	
	
	
	%%%%%%%%%%%%%%%%%%%%%%%%%%%%%%%%%%%%%%%%%%%%%%%%%%%%%%%%%%%%%
	%%% Problem 1
	
	
	\begin{itemize}
		\part{a} Draw a 2-dimensional range tree of minimal height for the following set of points:
		\begin{center}
			$\{(7, 88), (12, 19), (22, 33), (27, 29), (28, 9), (31, 99), (42, 66)\}$ 
		\end{center}
		
		\part{b} Suppose a two dimensional range tree data structure stores $n$ points, and that the BST ordered by $x$-coordinates is perfect, i.e., every level is completely filled.
		Give an exact closed form formula in terms of $n$ for the sum of the number of nodes in the $x$-ordered BST plus the total number of nodes in all $y$-ordered BSTs.
		
		\part{c} Assume that we have a set of $n$ numbers (not necessarily integers) and we are interested only in counting the number of points that lie in a range rather than in reporting all of them.
		Describe how a 1-dimensional range tree (i.e., a balanced BST) can be modified such that a range counting query can be performed in $O(\log n)$ time (independent of $s$).  
		Briefly justify that your algorithm is within the expected runtime.
		
		\part{d} Next, consider the 2-dimensional case where we have a set of $n$ 2-dimensional points.
		Given a query rectangle $R$, we only want to find the number of points inside $R$, not the points themselves.
		Explain how to modify the Range Tree data structure and the search algorithm such that counting queries can be performed in $O((\log n)^2))$ time.  
		Briefly justify that your algorithm meets the runtime requirement.
		
	\end{itemize}
	
	
	%%% END OF QUESTIONS %%%
\end{document}