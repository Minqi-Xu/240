\documentclass[12pt]{article}

%\usepackage{algo}
\usepackage{tikz,fullpage,url,amssymb,amsmath,epsfig,color,xspace,alltt,mathtools}
\usetikzlibrary{shapes,chains,positioning}
\usepackage[pdftitle={CS 240 Assignment 2},%
pdfsubject={University of Waterloo, CS 240, Fall 2021},%
pdfauthor={MP}]{hyperref}
%\RequirePackage{pstricks,pst-node,pst-tree} % draw trees, requires using xetex
\newlength{\nodeLength}
\newcommand{\Node}{A}
\newcommand{\setnode}[1]{
	\settowidth{\nodeLength}{#1}
	\renewcommand{\Node}[1]{
		\Tcircle[name=#1]{\makebox[\nodeLength]{##1}}
	}
}
\setnode{99}

\newcommand{\ceil}[1]{\left\lceil #1 \right\rceil}
\newcommand{\floor}[1]{\left\lfloor #1 \right\rfloor}
\renewcommand{\thesubsection}{Problem \arabic{subsection}}

\begin{document}
	
	\begin{center}
		{\Large\bf Problem 6}\\
		\vspace{3mm}
	\end{center}
	
	\definecolor{care}{rgb}{0,0,0}
	\def\question#1{\item[\bf #1.]}
	\def\part#1{\item[\bf #1)]}
	\newcommand{\pc}[1]{\mbox{\textbf{#1}}} % pseudocode
	
	
	
	%%%%%%%%%%%%%%%%%%%%%%%%%%%%%%%%%%%%%%%%%%%%%%%%%%%%%%%%%%%%%
	In quick sort, normally, we recursively call the quick sort and put 2 parts of the array as input.
	
	This time, we using  find-median which is $\Theta(n^2)$ to find pivot. And seperate the array into two equal length parts.
	
	Then we have $T(n) = \Theta(n^2) + 2T(\frac{n}{2})$
	
	This implies that $T(n) = \sum_{i=0}^{\log n} 2^i\cdot \Theta(n^2) = \Theta(n^2) \cdot \Theta(2^{\log n}) = \Theta(n^3)$
	
	
	%%%%%%%%%%%%%%%%%%%%%%%%%%%%%%%%%%%%%%%%%%%%%%%%%%%%%%%%%%%%%
\end{document}